\documentclass{article}
\usepackage{graphicx} % Required for inserting images
\usepackage{amsmath}
\usepackage{xcolor}
\begin{document}
\pagecolor {black}
\color{white}
\section{Today we will be solving this definite integral using Feynman's technique}

\Huge{\[\int_{0}^{\frac{\pi}{2}}ln(1+4sin^2(x))dx\]}
\\
\huge{
Define the following function
\[I(a)=\int_{0}^{\frac{\pi}{2}}ln(1+asin^2(x))dx\]
Note that our target integral is $I(4)$ and $I(0)=0$. To slightly simplify future steps, we will add the condition $a\ge0$.
\\
\\
Differentiate both sides of the equation with respect to a and apply the Leibniz Rule
\[I'(a)=\frac{d}{da}(\int_{0}^{\frac{\pi}{2}}ln(1+asin^2(x))dx)\]
\[I'(a)=\int_{0}^{\frac{\pi}{2}}\frac{\partial}{\partial a}(ln(1+asin^2(x)))dx\]
\[I'(a)=\int_{0}^{\frac{\pi}{2}}\frac{sin^2(x)}{1+asin^2(x)}dx\]
Apply the King's Property and use the fact that $sin(\frac{\pi}{2}-x)=cos(x)$
\[I'(a)=\int_{0}^{\frac{\pi}{2}}\frac{sin^2(\frac{\pi}{2}-x)}{1+asin^2(\frac{\pi}{2}-x)}dx\]
\[I'(a)=\int_{0}^{\frac{\pi}{2}}\frac{cos^2(x)}{1+acos^2(x)}dx\]
Divide all terms in the fraction by $cos^2(x)$
\[I'(a)=\int_{0}^{\frac{\pi}{2}}\frac{1}{sec^2(x)+a}dx\]
Use the trigonometric identity: $sec^2(x)=tan^2(x)+1$
\[I'(a)=\int_{0}^{\frac{\pi}{2}}\frac{1}{tan^2(x)+1+a}dx\]
Expand the fraction with $sec^2(x)$
\[I'(a)=\int_{0}^{\frac{\pi}{2}}\frac{sec^2(x)}{(tan^2(x)+1+a)(sec^2(x))}dx\]
\\
\\
\\
Use the trigonometric identity: $sec^2(x)=tan^2(x)+1$
\[I'(a)=\int_{0}^{\frac{\pi}{2}}\frac{sec^2(x)}{(tan^2(x)+1+a)(tan^2(x)+1)}dx\]
let $u=tan(x)$\\$du=sec^2(x)dx$\\$u(0)=0$\\
$\lim_{x\to(\frac{\pi}{2})^-} u(x)\to\infty$
\[I'(a)=\int_{0}^{\infty}\frac{1}{(u^2+1+a)(u^2+1)}du\]
Partial fraction decomposition
\[I'(a)=\int_{0}^{\infty}\frac{-1/a}{u^2+1+a}du+\int_{0}^{\infty}\frac{1/a}{u^2+1}du\]
\[I'(a)=\int_{0}^{\infty}\frac{-1/a}{u^2+(\sqrt{1+a})^2}du+\int_{0}^{\infty}\frac{1/a}{u^2+1^2}du\]
\[I'(a)=\frac{-1}{a}\int_{0}^{\infty}\frac{1}{u^2+(\sqrt{1+a})^2}du+\frac{1}{a}\int_{0}^{\infty}\frac{1}{u^2+1^2}du\]
\\
\\
Now we can integrate using the formula \[\int_{0}^{\infty}\frac{1}{u^2+b^2}du=\frac{\pi}{2b}\]
The proof of this formula is left as an
\\exercise for the viewer.
\[I'(a)=\frac{-1}{a}\frac{\pi}{2\sqrt{1+a}}+\frac{1}{a}\frac{\pi}{2}\]
\[I'(a)=\frac{\pi}{2a}-\frac{\pi}{2a\sqrt{1+a}}\]
\[I'(a)=\frac{\pi}{2}(\frac{1}{a}-\frac{1}{a\sqrt{1+a}})\]
Now we can integrate both sides of the equation with respect to $a$ to get $I(a)$ back. Since we are applying indefinite integrals to both sides, after this step the two sides of the equation may differ by a constant.
\[I(a)-C=\frac{\pi}{2}(\int\frac{1}{a}da-\int\frac{1}{a\sqrt{1+a}}da)\]
\[I(a)-C=\frac{\pi}{2}(ln(a)-\int\frac{1}{a\sqrt{1+a}}da)\]
Off to the side
\[B=\int\frac{1}{a\sqrt{1+a}}da\]
let $v^2=1+a$
\\$2vdv=da$
\\$v=\sqrt{1+a}$
\\$a=v^2-1$
\[B=\int\frac{2v}{(v^2-1)v}dv\]
\[B=\int\frac{2}{v^2-1}dv\]
\[B=\int\frac{2}{(v-1)(v+1)}dv\]
Partial fraction decomposition
\[B=\int\frac{1}{v-1}dv+\int\frac{-1}{v+1}dv\]
\[B=ln(v-1)-ln(v+1)\]
\[B=ln(\sqrt{1+a}-1)-ln(\sqrt{1+a}+1)\]
\[B=ln(\frac{\sqrt{1+a}-1}{\sqrt{1+a}+1})\]
Expand the fraction with the radical \\conjugate of the numerator
\[B=ln(\frac{(\sqrt{1+a}-1)(\sqrt{1+a}+1)}{(\sqrt{1+a}+1)(\sqrt{1+a}+1)})\]
\[B=ln(\frac{1+a-1}{(\sqrt{1+a}+1)^2})\]
\[B=ln(\frac{a}{(\sqrt{1+a}+1)^2})\]
\\
\[I(a)-C=\frac{\pi}{2}(ln(a)-ln(\frac{a}{(\sqrt{1+a}+1)^2}))\]
\[I(a)-C=\frac{\pi}{2}(ln(a)+ln(\frac{(\sqrt{1+a}+1)^2}{a}))\]
\[I(a)-C=\frac{\pi}{2}ln((\sqrt{1+a}+1)^2)\]
\[I(a)=\pi ln(\sqrt{1+a}+1)+C\]
Plugging in our condition from the start
\[I(0)=0=\pi ln(2)+C\]
\[\implies C=-\pi ln(2)\]
\[I(a)=\pi ln(\sqrt{1+a}+1)-\pi ln(2)\]
\[I(a)=\pi ln(\frac{1+\sqrt{1+a}}{2})\]\
\[I(4)=\int_{0}^{\frac{\pi}{2}}ln(1+4sin^2(x))dx\]
\[=\pi ln(\frac{1+\sqrt{5}}{2})\]
\[=\pi ln(\varphi)\]
}

\Huge{\[\int_{0}^{\frac{\pi}{2}}ln(1+4sin^2(x))dx=\pi ln(\varphi)\]}
\end{document}